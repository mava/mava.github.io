\documentclass[12pt]{article}

\author{Marco Varisco}
\title{Sage\TeX\ Example}
\date{Algebra/Topology Seminar, February 7, 2013}

\usepackage{sagetex}

\begin{document}

\maketitle
\thispagestyle{empty}

With Sage\TeX\ you can use Sage to compute and put stuff in your \LaTeX\ document.
Here are a couple silly examples.

\begin{itemize}

\item The first fifty decimal digits of~$\pi$ are:
\[
    \sage{pi.n(digits=51)}\ldots
\]

\begin{sagesilent}
    A = matrix(QQ, [[0, 1, 2], [1, 0, 3], [4, -3, 8]])
\end{sagesilent}
\item The inverse of the matrix $\sage{A}$ is~$\sage{A^-1}$.

\begin{sagesilent}
    R.<x,y,z> = PolynomialRing(QQ)
    B = matrix(R, [[1, x, y], [0, 1, z], [0, 0, 1]])
    latex.matrix_delimiters(left='[', right=']')
\end{sagesilent}
\item $\sage{A}\sage{B}=\sage{A*B}$.

\end{itemize}

\end{document}
